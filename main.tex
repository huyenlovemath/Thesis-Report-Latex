% \documentclass[12pt, a4paper]{article}
\documentclass[12pt,a4paper]{report}
\usepackage[utf8]{vietnam}
\usepackage[T5]{fontenc}

\usepackage{amsmath}
\usepackage{amsfonts}
\usepackage{amssymb}
\usepackage{makeidx}
\usepackage{imakeidx}
\usepackage{graphicx}
\usepackage{graphics}
\usepackage{placeins}
\usepackage[unicode, bookmarksopenlevel=4]{hyperref}
\usepackage{makeidx}
\usepackage{biblatex}
\usepackage{multicol}
\usepackage{subfiles}
\usepackage{hyperref}
\usepackage{enumitem}
\usepackage{float}
\usepackage[table,xcdraw]{xcolor}
\usepackage{tabularx}
\usepackage{wrapfig}
\usepackage{caption}
\usepackage{subcaption}
\usepackage{placeins}
\usepackage{array}
\usepackage{longtable}
\usepackage{multirow}
\usepackage{tikz}
\usepackage{pgfplots}
\usepackage{listings,xcolor}

\definecolor{dkgreen}{rgb}{0,.6,0}
\definecolor{dkblue}{rgb}{0,0,.6}
\definecolor{dkyellow}{cmyk}{0,0,.8,.3}

\usetikzlibrary{calc}

\let\orgautoref\autoref
\def\code#1{\texttt{#1}}

\setcounter{secnumdepth}{4}
\setcounter{tocdepth}{2}

\newcommand{\iindex}[1]{\textit{#1}\index{#1}}
\renewcommand{\lstlistingname}{Đoạn mã}% Listing -> Algorithm
\renewcommand{\lstlistlistingname}{Danh sách các đoạn mã}

% Create file reference.bib to add
\addbibresource{./reference.bib}

\graphicspath{ {./images/} {./../images}}
\DeclareGraphicsExtensions{.png,.eps,.svg}
\setlist[description]{leftmargin=\parindent,labelindent=\parindent}

\title{PHÁT TRIỂN CÔNG CỤ TRÍCH XUẤT VÀ ĐÁNH GIÁ ĐẶC TRƯNG PHỤC VỤ PHÁT HIỆN MÃ ĐỘC TRONG TÀI LIỆU PDF}

\lstset{
  language        = php,
  basicstyle      = \small\ttfamily,
  keywordstyle    = \color{dkblue},
  stringstyle     = \color{red},
  identifierstyle = \color{dkgreen},
  commentstyle    = \color{gray},
  emph            =[1]{php},
  emphstyle       =[1]\color{black},
  emph            =[2]{if,and,or,else},
  emphstyle       =[2]\color{dkyellow},
  breakatwhitespace =   false,
  breaklines    = true
  }

\pagenumbering{roman}
\begin{document}

\subfile{cover.tex}
\clearpage{}

\chapter*{Tóm tắt}

Trong những năm gần đây, PDF hay viết tắt của Portable Document Format, đã trở thành một định dạng tiêu chuẩn để lưu trữ và trao đổi tài liệu nhờ tính linh hoạt và những chức năng hữu ích. Việc sử dụng rộng rãi đã tạo ra những lầm tưởng về sự an toàn của tài liệu PDF với người dùng. Tuy nhiên, các đặc tính của PDF đã thúc đẩy tin tặc khai thác nhiều loại lỗ hổng khác nhau, vượt qua các biện pháp bảo vệ an toàn, từ đó khiến tài liệu PDF trở thành một trong những phương tiện tấn công mã độc hiệu quả, trong các cuộc tấn công quy mô lớn đến các cuộc tấn công có chủ đích. Do đó việc phát hiện các tệp PDF độc hại là rất quan trọng trong bảo mật thông tin. Một số kỹ thuật phân tích sẽ được đề xuất trong tài liệu, để trích xuất các đặc trưng giá trị cho phép phân biệt phần mềm độc hại hay lành tính. Từ những kỹ thuật đánh giá đặc trưng thông thường có thể gặp nhiều hạn chế, các kỹ thuật dựa trên học máy đã được áp dụng phục vụ việc phát hiện tự động tệp PDF độc hại từ một tập các mẫu huấn luyện. Mô hình đã được chứng minh hiệu quả hơn khi kết hợp nhiều đặc trưng khác nhau. Tuy nhiên, bản thân mô hình này phải đối mặt với nhiều thách thức khi gặp phải các cuộc tấn công né tránh - khi mà những mẫu độc hại được biến đổi trông lành tính. Từ đó ở phần sau, khóa luận giới thiệu một thuật toán học máy mục tiêu tạo nên mô hình phân loại không có dương tính giả, đảm bảo mang đến một tường lửa lọc những tệp chắc chắn độc hại, còn những tệp sạch sẽ được xử lý nâng cao ở những giai đoạn tiếp theo.


\chapter*{Lời cảm ơn}


Lời đầu tiên cho phép em được gửi lời cảm ơn tới Khoa Công Nghệ Thông Tin - Trường Đại học Công
Nghệ ĐHQG Hà Nội đã tạo điều kiện thuận lợi cho em được học tập, nghiên cứu và thực hiện
đề tài tốt nghiệp này.

Em cũng xin được bày tỏ lòng biết ơn sâu sắc tới thầy Lê Đình Thanh đã tận tình hướng dẫn,
đóng góp những ý kiến giúp em hoàn thành khóa luận tốt nghiệp một cách tốt nhất.

Em cũng vô cùng biết ơn các thầy cô trong trường tận tình giảng dạy, truyền thụ cho
em những kiến thức và kỹ năng quan trọng làm hành trang vững chắc trên con đường học vấn
của bản thân.

Chúc mọi người luôn luôn mạnh khoẻ và gặt hái được nhiều thành công trong cuộc sống.

\chapter*{Lời cam đoan}

Em xin cam đoan rằng khóa luận tốt nghiệp này không sao chép từ bất kỳ ai,
tổ chức nào khác. Toàn bộ nội dung được trình bày trong tài liệu đều là cá nhân em qua quá
trình học tập, tìm hiểu và nghiên cứu mà hoàn thiện. Mọi tài liệu tham khảo đều được ghi chép
lại và trích dẫn hợp pháp. Nếu lời cam đoan là sai sự thật thì em xin chịu mọi trách nhiệm và
hình thức kỷ luật theo quy định từ phía nhà trường.

\tableofcontents{}
\clearpage{}

\listoffigures{}

\listoftables{}

\lstlistoflistings

\chapter*{Mở đầu}
\pagenumbering{arabic}

\subfile{./sections/section_1.tex}

\chapter{Kiến thức nền tảng}

\subfile{./sections/section_2.tex}

\chapter{Kết hợp đặc trưng thống kê và cấu trúc cây trong phát hiện PDF độc hại}
\subfile{./sections/section_3.tex}

\chapter{Xây dựng mô hình phân loại Zero False Positive}

\subfile{./sections/section_4.tex}


\chapter{Kết luận}
\subfile{./sections/section_5.tex}


\appendix
\chapter{Phụ lục chứng minh}
\subfile{./sections/section_6.tex}



\nocite{*}

\printbibliography[heading=bibintoc, title=Tài liệu tham khảo]

% \chapter*{Từ điển chú giải}

\end{document}

