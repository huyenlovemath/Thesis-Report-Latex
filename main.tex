% \documentclass[12pt, a4paper]{article}
\documentclass[12pt,a4paper]{report}
\usepackage[utf8]{vietnam}
\usepackage[T5]{fontenc}

\usepackage{amsmath}
\usepackage{amsfonts}
\usepackage{amssymb}
\usepackage{makeidx}
\usepackage{imakeidx}
\usepackage{graphicx}
\usepackage{graphics}
\usepackage{placeins}
\usepackage[unicode, bookmarksopenlevel=4]{hyperref}
\usepackage{makeidx}
\usepackage[style=numeric]{biblatex} 
\usepackage{multicol}
\usepackage{subfiles}
\usepackage{hyperref}
\usepackage{enumitem}
\usepackage{float}
\usepackage[table,xcdraw]{xcolor}
\usepackage{tabularx}
\usepackage{wrapfig}
\usepackage{caption}
\usepackage{subcaption}
\usepackage{placeins}
\usepackage{array}
\usepackage{longtable}
\usepackage{multirow}
\usepackage{tikz}
\usepackage{pgfplots}
\usepackage{listings,xcolor}
\usepackage{glossaries}

\definecolor{dkgreen}{rgb}{0,.6,0}
\definecolor{dkblue}{rgb}{0,0,.6}
\definecolor{dkyellow}{cmyk}{0,0,.8,.3}

\usetikzlibrary{calc}

\let\orgautoref\autoref
\def\code#1{\texttt{#1}}

\setcounter{secnumdepth}{4}
\setcounter{tocdepth}{2}

% \newcommand{\iindex}[1]{\textit{#1}\index{#1}}
% \renewcommand{\lstlistingname}{Đoạn mã}% Listing -> Algorithm
% \renewcommand{\lstlistlistingname}{Danh sách các đoạn mã}

% Create file reference.bib to add
\addbibresource{./reference.bib}

\graphicspath{ {./images/} {./../images}}
\DeclareGraphicsExtensions{.png,.eps,.svg}
\setlist[description]{leftmargin=\parindent,labelindent=\parindent}

\title{PHÁT TRIỂN CÔNG CỤ TRÍCH XUẤT VÀ ĐÁNH GIÁ ĐẶC TRƯNG PHỤC VỤ PHÁT HIỆN MÃ ĐỘC TRONG TÀI LIỆU PDF}

\lstset{
  language        = php,
  basicstyle      = \small\ttfamily,
  keywordstyle    = \color{dkblue},
  stringstyle     = \color{red},
  identifierstyle = \color{dkgreen},
  commentstyle    = \color{gray},
  emph            =[1]{php},
  emphstyle       =[1]\color{black},
  emph            =[2]{if,and,or,else},
  emphstyle       =[2]\color{dkyellow},
  breakatwhitespace =   false,
  breaklines    = true
  }

\pagenumbering{roman}
\begin{document}

\subfile{cover.tex}
\clearpage{}

\chapter*{Tóm tắt}
Trong những năm gần đây, PDF hay viết tắt của Portable Document Format, là một định dạng lưu trữ tài liệu được sử dụng rộng rãi trên các nền tảng Internet. Tính linh hoạt trong cấu trúc và khả năng cho phép nhúng những mã thực thi như JavaScript, shellcode của PDF đã thúc đẩy tin tặc sử dụng tài liệu PDF như một phương tiện tấn công mã độc hiệu quả. Do đó việc phát hiện các tệp PDF độc hại là rất quan trọng trong bảo mật thông tin.

Hiện nay, phương pháp được sử dụng phổ biến trong phát hiện tài liệu PDF độc hại là trích xuất đặc trưng và sử dụng các mô hình phân loại học máy. Hiệu quả của các mô hình phân loại phụ thuộc rất lớn vào các tập đặc trưng. Vậy nên mục tiêu của khóa luận là tìm ra những tập đặc trưng tốt nhất để có thể xây dựng được một mô hình phát hiện tệp PDF độc hại hiệu quả.

Các đặc trưng có thể được phân thành 3 nhóm, bao gồm các đặc trưng thống kê, đặc trưng cấu trúc và đặc trưng JavaScript. Trong các nghiên cứu trước đây thường chỉ tập trung vào một nhóm đặc trưng cụ thể, ngược lại, khóa luận này sẽ tiến hành kết hợp cả 3 nhóm. Đầu tiên các đặc trưng cấu trúc và thống kê được kết hợp với nhau, xếp hạng và lựa chọn để tạo ra một mô hình phân loại hiệu quả hơn các mô hình đã có chỉ sử dụng 1 nhóm đặc trưng. Tiếp theo, đối với đặc trưng thứ ba, bởi mã JavaScript thường được làm rối khiến cho các mô hình phân lớp khó có thể phân biệt được mã lành tính với mã độc hại, nên khóa luận đưa ra một cách xử lý là chia quá trình phát hiện thành 2 giai đoạn. Giai đoạn 1 sử dụng mô hình phân loại chỉ sử dụng hai nhóm đặc trưng đầu với mục tiêu sàng lọc những tệp PDF độc không chứa JavaScript hoặc chứa JavaScript dễ phát hiện. Giai đoạn 2, với những tệp khó phát hiện, mã JavScript được phân tích và xử lý bằng những kỹ thuật chuyên sâu nhưng tốn kém hơn. Theo phương pháp xử lý này, khóa luận đã điều chỉnh mô hình phân lớp ở giai đoạn một để đảm bảo không cho ra dương tính giả. Thuật toán được sử dụng là thuật toán lặp Zero False Positive, được đảm bảo tính đúng đắn thông qua chứng minh và thử nghiệm trên tập huấn luyện.

\chapter*{Lời cảm ơn}


Đầu tiên tôi xin được gửi lời cảm ơn chân thành tới thầy Lê Đình Thanh đã chỉ dẫn tận tình và đóng góp ý kiến giúp tôi có thể hoàn thành khóa luận tốt nghiệp một cách trọn vẹn nhất.

Tôi cũng xin bày tỏ lòng biết ơn tới các thầy cô cùng Khoa Công Nghệ Thông Tin - Trường Đại học Công
Nghệ ĐHQG Hà Nội truyền thụ cho tôi những kiến thức quan trọng, đồng thời tạo điều kiện cho tôi được nghiên cứu và thực hiện đề tài tốt nghiệp này.

\chapter*{Lời cam đoan}

Tôi xin cam đoan rằng khóa luận "Phát triển công cụ trích xuất và đánh giá đặc trưng phục vụ phát hiện mã độc trong tài liệu PDF" này không sao chép từ bất kỳ ai,
tổ chức nào khác. Toàn bộ nội dung trong tài liệu là cá nhân tôi qua quá
trình học tập và nghiên cứu mà hoàn thiện. Mọi tài liệu tham khảo đều được trích dẫn hợp pháp. Nếu lời cam đoan là sai sự thật thì tôi xin chịu mọi trách nhiệm và
hình thức kỷ luật theo quy định từ phía nhà trường.

\tableofcontents{}
\clearpage{}

\listoffigures{}

\listoftables{}

% \clearpage{}
% \printglossaries[title=Danh sách thuật ngữ]
\clearpage

\chapter*{Mở đầu}
\pagenumbering{arabic}

\subfile{./sections/section_1.tex}

\chapter{Kiến thức nền tảng}

\subfile{./sections/section_2.tex}

\chapter{Kết hợp đặc trưng thống kê và cấu trúc cây trong phát hiện PDF độc hại}

\subfile{./sections/section_3.tex}

\chapter{Xây dựng mô hình phân loại Zero False Positive}

\subfile{./sections/section_4.tex}


\chapter{Kết luận}
\subfile{./sections/section_5.tex}


\appendix
\chapter{Chứng minh tính đúng đắn của mô hình Zero False Positive}
\subfile{./sections/section_6.tex}



\nocite{*}

\printbibliography[heading=bibintoc, title=Tài liệu tham khảo]

% \chapter*{Từ điển chú giải}

\end{document}

