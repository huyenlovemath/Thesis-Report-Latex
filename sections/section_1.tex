% !TeX root = ../main.tex
\documentclass[./../main.tex]{subfiles}

\begin{document}
Ngày nay, những kẻ tấn công bằng nhiều cách có thể cài đặt các phần mềm độc hại trên máy nạn nhân để có được quyền truy cập, thực hiện các hành vi xâm phạm dữ liệu cá nhân, phá hoại hệ thống hoặc sử dụng vào những mục đích xấu khác. Người dùng thường có thể dễ dàng nhận ra mối đe dọa từ những tệp thực thi hay những tài liệu Microsoft Office. Tuy nhiên, bỏ qua sự phức tạp về định dạng cấu trúc tệp PDF, người dùng có xu hướng coi các tệp PDF là tài liệu vô hại.

Hiện nay, lợi dụng nhiều lỗ hổng bảo mật, các trình đọc tài liệu PDF đang là mục tiêu của các kẻ tấn công mạng. Điển hình với trình đọc PDF Acrobat Reader, từ năm 1999 tới nay, đã có tới 298 lỗ hổng được phát hiện theo thống kê của CVE Details (Hình \ref{fig:acrobatcve}).

% figure
\begin{figure}[ht!]
	\includegraphics[width=\linewidth]{./images/VulPerTime.png}
	\caption{Biểu đồ các lỗ hổng của trình đọc PDF Acrobat Reader theo các năm\protect\footnotemark}
	\label{fig:acrobatcve}
\end{figure}
\footnotetext{\url{https://www.cvedetails.com}}

Trong khóa luận này, tôi sẽ trình bày một số phương pháp để phát hiện các tệp PDF độc hại, trong đó các đặc trưng được trích xuất từ tài liệu PDF sẽ được sử dụng để xây dựng bộ phân loại học máy tự động phát hiện tài liệu PDF độc hại, thông qua các chương sau:
\begin{description}
	\item [Chương 1] Kiến thức nền tảng về định dạng tài liệu PDF, một số thuật toán học máy, từ đó đưa ra một quy trình để phát hiện phần mềm độc hại.
	\item [Chương 2] Thực hiện trích xuất đặc trưng trên tập dữ liệu gồm khoảng 22000 tệp PDF, từ đó xây dựng bộ phân loại học máy và đánh giá kết quả khi kết hợp các đặc trưng khác nhau.
	\item [Chương 3] Đề xuất phương hướng phát hiện tài liệu PDF độc hại thông qua hai giai đoạn, với giai đoạn đầu xây dựng một thuật toán học máy không gây phân loại dương sai và giai đoạn hai gồm các phương hướng xử lý nâng cao với các tệp được gán nhãn sạch.
	\item [Kết luận] Đánh giá kết quả đạt được, đưa ra những hạn chế và khó khăn gặp phải, cuối cùng đề xuất những phương hướng phát triển đề tài trong tương lai.
\end{description}


\end{document}
% 
% \footnote{\url{https://www.cvedetails.com}}