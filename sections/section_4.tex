% !TeX root = ../main.tex
\documentclass[./../main.tex]{subfiles}

\begin{document}
\section{Đặt vấn đề}
\section{Bộ phân loại Zero False Positive}
\subsection{Thuật toán phân loại lặp với mục tiêu FP = 0}
\subsection{Thực nghiệm và đánh giá kết quả}
Tiến hành thực nghiệm để kiểm định mô hình trên tập huấn luyện gồm các tệp PDF đã được trích xuất đặc trưng thống kê và đặc trưng cấu trúc trong chương 2. Kết quả cho thấy số lượng FP qua các vòng lặp đều được đảm bảo bằng 0 và số lượng FN giảm dần. Tuy nhiên, từ vòng lặp thứ 700 - 1000, kết quả FN không giảm, duy trì số lượng 97, với tỉ lệ FNR bằng 0.0095.

\begin{table}[]
	\centering
	\caption{Kết quả của thuật toán phân loại Zero False Positive trên tập dữ liệu đã trích xuất}
	\label{tab:ket_qua_thuat_toan_phan_loai}
	\begin{tabular}{|c|c|c|c|c|}
		\hline
		\textbf{Iteration} & \textbf{TN} & \textbf{FP} & \textbf{FN} & \textbf{TP} \\ \hline
		0                  & 7423        & 0           & 2993        & 7220        \\ \hline
		5                  & 7423        & 0           & 361         & 9852        \\ \hline
		10                 & 7423        & 0           & 255         & 9988        \\ \hline
		50                 & 7423        & 0           & 133         & 10080       \\ \hline
		100                & 7423        & 0           & 107         & 10106       \\ \hline
		200                & 7423        & 0           & 105         & 10108       \\ \hline
		500                & 7423        & 0           & 100         & 10113       \\ \hline
		700                & 7423        & 0           & 97          & 10116       \\ \hline
		1000               & 7423        & 0           & 97          & 10116       \\ \hline
	\end{tabular}
\end{table}

Ngoài ra, mô hình được tiến hành thử nghiệm trên một số tập dữ liệu khác. Bộ dữ liệu đặc trưng tệp PDF được trích xuất và công bố bởi đội ngũ phát triển Hidost [ref hidost reputation] với một số lượng lớn các mẫu PDF độc hại được thu thập qua nhiều năm. Tôi đã liên hệ Nedim Srndic, một trong những tác giả của Hidost để xin phép sử dụng bộ dữ liệu đặc trưng trong phạm vi khóa luận này. Bộ dữ liệu được phân thành nhiều tập, và dưới đây là kết quả của hai tập dữ liệu khi thử nghiệm trên mô hình phân loại Zero False Positive.

\begin{table}[]
	\centering
	\caption{Kết quả thuật toán phân loại Zero False Positive trên tập dữ liệu Hidost (1)}
	\label{tab:ket_qua_thuat_toan_phan_loai_hidost}
	\begin{tabular}{|c|c|c|c|c|}
		\hline
		\textbf{Iteration} & \textbf{TN} & \textbf{FP} & \textbf{FN} & \textbf{TP} \\ \hline
		0                  & 142796      & 0           & 1851        & 4406        \\ \hline
		10                 & 142796      & 0           & 410         & 5847        \\ \hline
		50                 & 142796      & 0           & 287         & 5970        \\ \hline
		100                & 142796      & 0           & 272         & 5985        \\ \hline
		120                & 142796      & 0           & 271         & 5986        \\ \hline
		160                & 142796      & 0           & 271         & 5986        \\ \hline
	\end{tabular}
\end{table}

\begin{table}[]
	\centering
	\caption{Kết quả thuật toán phân loại Zero False Positive trên tập dữ liệu Hidost (2)}
	\label{tab:ket_qua_thuat_toan_phan_loai_hidost_2}
	\begin{tabular}{|c|c|c|c|c|}
		\hline
		\textbf{iteration} & \textbf{TN} & \textbf{FP} & \textbf{FN} & \textbf{TP} \\ \hline
		0                  & 132465      & 0           & 4919        & 2484        \\ \hline
		10                 & 132465      & 0           & 539         & 6864        \\ \hline
		50                 & 132465      & 0           & 317         & 7086        \\ \hline
		100                & 132465      & 0           & 305         & 7098        \\ \hline
		190                & 132465      & 0           & 305         & 7098        \\ \hline
	\end{tabular}
\end{table}

Với số lượng tập sạch lớn, chênh lệch so với số lượng tập độc, tập dữ liệu của Hidost yêu cầu thời gian chạy lớn khi thực hiện thuật toán phân loại Zero False Positive, và tỉ lệ FNR còn khá cao, 0.043 với tập thứ 1 và 0.041 ở tập thứ 2.

\section{Phương hướng xử lý các tệp được gán nhãn sạch}
Sau khi lọc các tệp PDF có chứa mã javascript qua mô hình phân loại Zero False Positive, các tệp được gán nhãn sạch sẽ có nguy cơ chứa các tệp độc hại. Từ đây các mẫu này sẽ tiếp tục được xử lý tại giai đoạn hai: thực hiện phân tích và trích xuất những đặc trưng liên quan tới javascript được nhúng trong tài liệu PDF. Trong phần này, một số cách tiếp cận xử lý nâng cao bao gồm cả phân tích tĩnh và phân tích động sẽ được giới thiệu.
\end{document}

