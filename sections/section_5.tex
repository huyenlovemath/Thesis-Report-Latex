% !TeX root = ../main.tex
\documentclass[./../main.tex]{subfiles}


\begin{document}

\section{Kết quả đạt được}
Khóa luận đã trình bày những phương pháp phân tích, trích xuất đặc trưng và đề xuất các mô hình sử dụng trong phân loại tự động các tài liệu PDF độc hại.

% chương 2
Thứ nhất, khóa luận đã chỉ ra rằng, bằng cách trích xuất một tập hợp các đặc trưng trên phạm vi rộng, kết hợp nhiều loại đặc trưng khác nhau có thể tạo ra một mô hình phát hiện tệp PDF độc hại mạnh mẽ, mang lại độ chính xác cao.
% tóm tắt những gì đã làm và kết quả đạt được ở chương 2
Cụ thể, khóa luận đã tiến hành trích xuất các đặc trưng ..., -> kết quả tốt hơn khi nếu chỉ sử dụng từng nhóm đặc trưng đơn lẻ.

% chương 3
Thứ hai, khóa luận đã điều chỉnh bộ phân lớp để đảm bảo không đưa ra dương tính giả. Bộ phân lớp sau khi điều chỉnh đóng vai trò sàng lọc các tệp PDF độc hại. Những tệp PDF độc hại khó phát hiện (âm tính giả), do có chứa mã JavaScript đã làm rối, được đề xuất chuyển sang giai đoạn xử lý chuyên sâu và tốn kém hơn.
Khi những tài liệu độc hại sử dụng nhiều kỹ thuật nâng cao, tinh vi nhằm che giấu những hành vi độc hại và bắt chước tính bình thường của các tệp sạch trong tập huấn luyện, thì chỉ sử dụng những đặc trưng thống kê và cấu trúc sẽ không thực sự hiệu quả, gây ra một số lượng âm tính giả cao. Từ đó, với việc đề xuất một phương pháp phát hiện tài liệu độc hại với hai giai đoạn, ở giai đoạn một, đã chỉ ra được một mô hình có thể mang lại tỷ lệ dương tính giả (FP) bằng 0, trong khi tỷ lệ âm tính giả (FN) được giữ ở mức tối thiểu trên tập huấn luyện. Mục tiêu này đã được thử nghiệm trên bộ dữ liệu được trích xuất ở thí nghiệm 1 và một số bộ dữ liệu đặc trưng được trích xuất trong một số công bố khác. Mô hình này được ứng dụng trong phạm vi khóa luận như một tường lửa, bước đầu lọc ra những tài liệu đảm bảo độc hại, trong khi những tài liệu được gán nhãn sạch sẽ tiếp tục được xử lý qua giai đoạn hai, thực hiện phân tích sâu hơn về các đoạn mã nhúng trong tài liệu.


\section{Hạn chế và khó khăn}

Trong quá trình cài đặt các công cụ trích xuất đặc trưng, tôi đã gặp những khó khăn:
\begin{itemize}
	\item
	      Các công cụ hỗ trợ trích xuất đặc trưng khá cũ, yêu cầu cài đặt môi trường và một số gói phụ thuộc (package dependencies) cụ thể, gây khó khăn trong việc tiến hành nhiệm vụ trích xuất đặc trưng. Ngoài ra nhiều tệp PDF có cấu trúc biến đổi, gây khó khăn cho một số trình phân tích cú pháp.
	\item
	      Các đặc trưng API JavaScript đã được trích xuất nhưng số lượng ít và không mang lại kết quả khả quan nên không được đề cập tới trong thí nghiệm của khóa luận này.

\end{itemize}
Trong quá trình xây dựng mô hình và huấn luyện, với tập dữ liệu khá lớn, cùng với quá trình điều chỉnh tham số và chọn mô hình tốt nhất đòi hỏi một thời gian chạy dài. Mô hình Zero False Positive cũng cần một số vòng lặp lớn để có thể mang lại kết quả có ý nghĩa.

\section{Định hướng phát triển}
\end{document}