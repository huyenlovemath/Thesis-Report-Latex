% !TeX root = ../main.tex
\documentclass[./../main.tex]{subfiles}


\begin{document}

\section{Kết quả đạt được}
Khóa luận đã trình bày những phương pháp phân tích, trích xuất đặc trưng và đề xuất các mô hình sử dụng trong phân loại tự động các tài liệu PDF độc hại.

Thứ nhất, khóa luận đã chỉ ra rằng, bằng cách trích xuất một tập hợp các đặc trưng trên phạm vi rộng, kết hợp nhiều loại đặc trưng khác nhau có thể tạo ra một mô hình phát hiện PDF độc hại mạnh mẽ và mang lại độ chính xác cao. Cụ thể, bằng việc sử dụng các công cụ PeePDF, PDFiD hay Hidost, những đặc trưng tĩnh đã được trích xuất bao gồm các đặc trưng có tính thống kê như từ khóa, metafeatures,... và các đặc trưng đường dẫn cấu trúc cây tài liệu PDF. Kết quả thí nghiệm đã chỉ ra rằng mô hình học máy mang lại kết quả tốt hơn khi tăng dần số lượng đặc trưng lựa chọn từ từng tập đặc trưng đã trích xuất. Bên cạnh đó, khi kết hợp hai tập đặc trưng thống kê và đặc trưng cấu trúc, mô hình đã cho thấy hiệu quả hơn khi chỉ sử dụng một loại đặc trưng, điều này được kiểm chứng bằng kết quả phân loại trên tập dữ liệu đã chuẩn bị.

Thứ hai, khóa luận đã đề xuất một phương pháp phát hiện tài liệu độc hại theo hai giai đoạn, với mục tiêu có thể xử lý tốt các tệp PDF có chứa những mã Javascript. Nhận thấy những tài liệu độc hại thường sử dụng nhiều kỹ thuật làm rối nâng cao, tinh vi nhằm che giấu những hành vi độc hại của các đoạn mã JavaScript và bắt chước tính bình thường của các tệp lành tính, thì chỉ sử dụng những đặc trưng thống kê và cấu trúc sẽ không thực sự hiệu quả, gây ra một số lượng âm tính giả cao. Từ đó, phương pháp chia hai giai đoạn được sử dụng để giải quyết vấn đề này. Ở giai đoạn một, khóa luận đã chỉ ra được một mô hình có thể mang lại tỷ lệ dương tính giả (FP) bằng 0, trong khi tỷ lệ âm tính giả (FN) được giữ ở mức tối thiểu trên tập huấn luyện.  Mô hình này được ứng dụng trong phạm vi khóa luận như một tường lửa, bước đầu lọc ra những tài liệu đảm bảo độc hại, trong khi những tài liệu được gán nhãn sạch sẽ tiếp tục được xử lý qua giai đoạn hai, thực hiện phân tích sâu hơn về các đoạn mã nhúng trong tài liệu. Mục tiêu này đã được kiểm chứng trên bộ dữ liệu với các đặc trưng thống kê và cấu trúc cây đã trích xuất ở thử nghiệm chương 2. Ngoài ra một số bộ dữ liệu đặc trưng đường dẫn cấu trúc cây được trích xuất trong công bố của Hidost \cite{hidost} cũng được sử dụng để chứng minh tính đúng đắn của thuật toán Zero False Positive trên. Ở giai đoạn hai, khóa luận đã đề xuất các phương phương phát triển trong tương lai để xử lý các đoạn mã JavaScript bị làm rối, từ đó có thể hoàn toàn lọc được các tệp PDF độc hại còn sót lại.


\section{Hạn chế và khó khăn}

Trong quá trình cài đặt các công cụ trích xuất đặc trưng, tôi đã gặp những khó khăn:
\begin{itemize}
	\item
	      Các công cụ hỗ trợ trích xuất đặc trưng khá cũ, yêu cầu cài đặt môi trường và một số gói phụ thuộc (package dependencies) cụ thể, gây khó khăn trong việc tiến hành nhiệm vụ trích xuất đặc trưng. Ngoài ra nhiều tệp PDF có cấu trúc biến đổi, gây khó khăn cho một số trình phân tích cú pháp.
	\item
	      Các đặc trưng API JavaScript đã được trích xuất nhưng số lượng ít và không mang lại kết quả khả quan nên không được đề cập tới trong thí nghiệm của khóa luận này.

\end{itemize}
Trong quá trình xây dựng mô hình và huấn luyện, với tập dữ liệu khá lớn, cùng với quá trình điều chỉnh tham số và chọn mô hình tốt nhất đòi hỏi một thời gian chạy dài. Bên cạnh đó, mô hình False Zero Positive đảm bảo không có dương tính giả trên tập huấn luyện, để đảm bảo kết quả này trên tập kiểm thử, cần phải xem xét ngưỡng dừng của tỉ lệ FNR ở mức không quá nhỏ hoặc cần có những cải tiến trong xây dựng mô hình.

\section{Định hướng phát triển}

Về định hướng phát triển đề tài trong tương lai, tôi đề xuất một số phương pháp phân tích và trích xuất được các đặc trưng JavaScript có giá trị để phục vụ phát hiện các tệp PDF độc hại.
\begin{itemize}
	\item Sử dụng công cụ PhoneyPDF để trích xuất các API JavaScript tương tác với trình đọc PDF Acrobat Reader bằng cả phân tích tĩnh và phân tích động. Từ đó có thể có được một tập các đặc trưng gồm các API thể hiện hành vi của các đoạn mã javascript được nhúng trong tệp PDF \cite{luxor}.
	\item Sử dụng trình biên dịch JavaScript của Google là V8 để trích xuất các bytecode từ Javascript- dưới dạng mã máy, mục tiêu có thể chống lại sự ẩn giấu hành vi của các đoạn mã JavaScript bị làm rối. Rozi và các cộng sự \cite{bytecode} đã sử dụng phương pháp phân tích này để trích xuất ra được tập các đặc trưng là các chuỗi bytecode, sau đó sử dụng một mô hình học sâu DPCNN \cite{dpcnn} để phân loại và phát hiện các tài liệu PDF độc hại.
	\item Một hướng phân tích tĩnh các đoạn mã JavaScript cũng hiệu quả được Vatamanu và cộng sự \cite{vatamanu} đề xuất, sử dụng công cụ SpiderMonkey để trích xuất ra các đặc trưng JS Token - một dạng trừu tượng hóa mã JavaScript nhằm loại bỏ những chi tiết gây rối giúp dễ dàng nhận diện và phát hiện bất thường.
\end{itemize}
\end{document}